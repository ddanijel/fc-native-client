\chapter*{Abstract}
\addcontentsline{toc}{chapter}{Abstract}

\selectlanguage{english}

While traditional supply chain (SC) systems met their requirements, they eventually reached their limits especially in a sense they do not provide enough transparency for the final consumers. They are also highly specialized in supplying just specific goods, meaning they are not generic enough to be reused for other types of goods. This is especially true for food, which safety and transparency are paramount to the final consumers. Traditional SC systems require a lot of manual work which leaves room for errors and manipulating with goods. Also, traditional systems lack integration between different systems in the supply chain which requires more work in order to make those systems work together. The commence of the Internet changed the way how traditional SC systems functioned. The more agile and dynamic way of working was introduced. That allowed the consumers to interact with their products at any of its stages. While the traditional SC systems have been improved over the course of time, there is room to advance the existing SC systems to the next level, especially when speaking about the transparency and making them more user-friendly for the final consumers. In this independent study, the aforementioned requirements are addressed by developing the application that can be used by both consumers and producers. 




\selectlanguage{english}
